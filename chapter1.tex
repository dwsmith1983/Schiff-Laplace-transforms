\chapter{Basic Principles}

\section{The Laplace Transform}

\begin{exercise}
\item
  From the definition of the Laplace transform compute
  \(\mathcal{L}\{f(t)\}(s)\)
  for
  \begin{exercise}[label = (\alph*), ref = \arabic{exercisei} (\alph*)]
  \item
    \(f(t) = 4t\)
    \par\smallskip
    The Laplace transform is defined as
    \(\mathcal{L}\{f(t)\}(s) = \int_0^{\infty}f(t)e^{-st}dt\).
    Using the definition, we have
    \[
    \mathcal{L}\{4t\}(s) = \int_0^{\infty}4te^{-st}dt =
    4\lim_{\tau\to\infty}\int_0^{\tau}te^{-st}dt
    \]
    We can now use integration by parts or we may notice that
    \(-\frac{\partial}{\partial s}e^{-st} = te^{-st}\).
    Note that \(t\) and \(e^{-st}\) are entire functions in both the variable
    \(s\) and \(t\); that is, \(t\) and \(e^{-st}\) can be represented by a
    power series that converges everywhere in the complex plane with an
    infinite radius of convergence.
    In order to justify differentiation under the integral sign, the following
    theorem must be satisfied.
    \par\smallskip
    \textbf{Theorem:} Let \(f(s,t)\) be a function such that both \(f(s,t)\)
    and its partial derivatives are continuous in \(s\) and \(t\) in some
    region of the \((s,t)\)-plane, including \(a(s)\leq t\leq b(s)\),
    \(s_0\leq s\leq s_1\).
    Also suppose that the functions \(a(s)\) and \(b(s)\) are both continuous
    and both have continuous derivatives for \(s_0\leq s\leq s_1\).
    Then for \(s_0\leq s\leq s_1\)
    \[
    \frac{d}{ds}\int_{a(s)}^{b(s)}f(s, t)dt = f(s, b(s))b'(s) - f(s,(a(s))a'(s)
    + \int_{a(s)}^{b(s)}f_s(s, t)dt.
    \]
    Since \(f(s,t) = te^{-st}\) which is the product of two entire functions
    so is entire and \(t,e^{-st}\in C^{\infty}\), the theorem is satisfied.
    Therefore, we can write
    \begin{align*}
      4\lim_{\tau\to\infty}\int_0^{\tau}te^{-st}dt
      & = -4\frac{d}{ds}\lim_{\tau\to\infty}\int_0^{\tau}e^{-st}dt\\
      & = 4\frac{d}{ds}\lim_{\tau\to\infty}\frac{e^{-st}}{s}\Bigr|_0^{\tau}\\
      & = 4\frac{d}{ds}\lim_{\tau\to\infty}
        \Bigl(\frac{e^{-s\tau} - 1}{s}\Bigr)\\
      & = \frac{d}{ds}\frac{-4}{s}\\
      \mathcal{L}\{4t\}(s) & = \frac{4}{s^2}
    \end{align*}
  \item
    \(f(t) = e^{2t}\)
    \par\smallskip
    For brevity, I am going to drop \(\lim_{\tau\to\infty}\int_0^{\tau}\) and
    just write \(\int_0^{\infty}\).
    \[
    \mathcal{L}\{e^{2t}\}(s) = \int_0^{\infty}e^{t(2 - s)}dt =
    \frac{e^{t(2 - s)}}{2 - s}\biggr|_0^{\infty} = \frac{1}{s - 2}
    \]
    A justifiable question would be why don't we have \(\infty\) since
    \(\lim_{t\to\infty}e^{t(2 - s)} = \infty\).
    If this was the case, we would have a divergent integral.
    In order for our integral to converge, we have to mind the exponential
    term.
    Let \(s = \sigma + i\omega\).
    Then \(\exp[2t - \sigma]\exp[-i\omega]\).
    \[
    \biggl\lvert\int_0^{\infty}\exp[(2 - \sigma)t]\exp[-it\omega]dt\biggr\rvert
    \leq\int_0^{\infty}\bigl\lvert\exp[(2 - \sigma)t]\bigr\rvert
    \bigl\lvert\exp[-it\omega]\bigr\rvert dt
    \]
    By Euler's formula, \(\exp[-it\omega] = \cos(t\omega) - i\sin(t\omega)\) so
    \(\lvert e^{-i\omega}\rvert =
    \sqrt{\cos^2(t\omega) + \sin^2(t\omega)} = 1\).
    Thus, we have
    \[
    \int_0^{\infty}\exp[t(2 - \sigma)]dt
    \]
    which converges when \(t(2 - \sigma) < 0\iff 2 < \sigma = \Re\{s\}\).
    Therefore, since \(t(2 - \sigma) < 0\), we are justified in writing
    \(-t(\sigma - 2)\) since \(\sigma - 2 > 0\).
    We then have \(\mathcal{L}\{e^{2t}\}(s) = \frac{1}{s - 2}\).
  \item
    \label{1c}
    \(f(t) = 2\cos(3t)\)
    \par\smallskip
    For this problem, we will need to use integration by parts twice.
    Let \(u = e^{-st}\) and \(dv = \cos(3t)dt\).
    Then \(du = -se^{-st}\) and \(v = \sin(3t)/3\)
    \begin{align*}
      \mathcal{L}\{2\cos(3t)\}(s)
      & = 2\int_0^{\infty}\cos(3t)e^{-st}dt\\
      & = \cancelto{0}{\frac{2e^{-st}\sin(3t)}{3}\biggl|_0^{\infty}} +
        \frac{2s}{3}\int_0^{\infty}\sin(3t)e^{-st}dt\\
      \intertext{For the second integration by parts, \(u\) and \(du\) will
      remain the same but \(dv = \sin(3t)dt\) so \(v = -\cos(3t)/3\).}
      & = \frac{2s}{3}\Biggl[\frac{-e^{-st}\cos(3t)}{3}\biggl|_0^{\infty} -
        \frac{s}{3}\int_0^{\infty}\cos(3t)e^{-st}dt\Biggr]\\
      (1 + s^2/9)2\int_0^{\infty}\cos(3t)e^{-st}dt & = \frac{2s}{9}\\
      \mathcal{L}\{2\cos(3t)\}(s) & = \frac{2s}{s^2 + 9}
    \end{align*}
  \item
    \(f(t) = 1 - \cos(\omega t)\)
    \par\smallskip
    By \cref{1c}, we have
    \[
    \mathcal{L}\{1 - \cos(\omega t)\}(s) = \int_0^{\infty}e^{-st}dt -
    \int_0^{\infty}\cos(\omega t)e^{-st}dt =
    \frac{1}{s} - \frac{s}{s^2 + \omega^2}
    \]
  \item
    \(f(t) = te^{2t}\)
  \item
    \(f(t) = e^t\sin(t)\)
  \item
    \(f(t) =
    \begin{cases}
      1, & t\geq a\\
      0, & t < 0
    \end{cases}\)
  \item
    \(f(t) =
    \begin{cases}
      \sin(\omega t), & 0 < t < \pi/\omega\\
      0, & \pi/\omega\leq t
    \end{cases}\)
  \item
    \(f(t) =
    \begin{cases}
      2, & t\leq 1\\
      e^t, & t > 1
    \end{cases}\)
  \end{exercise}
\item
  Compute the Laplace transform of the function \(f(t)\) whose graph is given
  in \cref{ch1sec1}.
  \begin{figure}[H]
    \centering
    \subcaptionbox{}{
      \includestandalone[width = 3in, mode = image]{Tikz/ch1sec1prob2a}}
    \qquad
    \subcaptionbox{}{
      \includestandalone[width = 3in, mode = image]{Tikz/ch1sec1prob2b}}
    \caption{Plots of \(f(t)\) for problem two.}
    \label{ch1sec1}
  \end{figure}
\end{exercise}

%%% Local Variables:
%%% mode: latex
%%% TeX-master: t
%%% End:
