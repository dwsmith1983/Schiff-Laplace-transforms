\chapter{Basic Principles}

\section{The Laplace Transform}

\begin{exercise}
\item
  From the definition of the Laplace transform compute
  \(\mathcal{L}\{f(t)\}(s)\)
  for
  \begin{exercise}[label = (\alph*), ref = \arabic{exercisei} (\alph*)]
  \item
    \(f(t) = 4t\)
    \par\smallskip
    The Laplace transform is defined as
    \(\mathcal{L}\{f(t)\}(s) = \int_0^{\infty}f(t)e^{-st}dt\).
    Using the definition, we have
    \[
  \mathcal{L}\{4t\}(s) = \int_0^{\infty}4te^{-st}dt =
  4\lim_{\tau\to\infty}\int_0^{\tau}te^{-st}dt
\]
We can now use integration by parts or we may notice that
\(-\frac{\partial}{\partial s}e^{-st} = te^{-st}\).
Note that \(t\) and \(e^{-st}\) are entire functions in both the variable
\(s\) and \(t\); that is, \(t\) and \(e^{-st}\) can be represented by a
power series that converges everywhere in the complex plane with an
infinite radius of convergence.
In order to justify differentiation under the integral sign, the following
theorem must be satisfied.
\par\smallskip
\textbf{Theorem:} Let \(f(s,t)\) be a function such that both \(f(s,t)\)
and its partial derivatives are continuous in \(s\) and \(t\) in some
region of the \((s,t)\)-plane, including \(a(s)\leq t\leq b(s)\),
\(s_0\leq s\leq s_1\).
Also suppose that the functions \(a(s)\) and \(b(s)\) are both continuous
and both have continuous derivatives for \(s_0\leq s\leq s_1\).
Then for \(s_0\leq s\leq s_1\)
\[
  \frac{d}{ds}\int_{a(s)}^{b(s)}f(s, t)dt = f(s, b(s))b'(s) - f(s,(a(s))a'(s)
  + \int_{a(s)}^{b(s)}f_s(s, t)dt.
\]
Since \(f(s,t) = te^{-st}\) which is the product of two entire functions
so is entire and \(t,e^{-st}\in C^{\infty}\), the theorem is satisfied.
Therefore, we can write
\begin{align*}
  4\lim_{\tau\to\infty}\int_0^{\tau}te^{-st}dt
  & = -4\frac{d}{ds}\lim_{\tau\to\infty}\int_0^{\tau}e^{-st}dt\\
  & = 4\frac{d}{ds}\lim_{\tau\to\infty}\frac{e^{-st}}{s}\Bigr|_0^{\tau}\\
  & = 4\frac{d}{ds}\lim_{\tau\to\infty}
    \Bigl(\frac{e^{-s\tau} - 1}{s}\Bigr)\\
  & = \frac{d}{ds}\frac{-4}{s}\\
  \mathcal{L}\{4t\}(s) & = \frac{4}{s^2}
\end{align*}
\item
  \(f(t) = e^{2t}\)
  \par\smallskip
  For brevity, I am going to drop \(\lim_{\tau\to\infty}\int_0^{\tau}\) and
  just write \(\int_0^{\infty}\).
  \[
  \mathcal{L}\{e^{2t}\}(s) = \int_0^{\infty}e^{t(2 - s)}dt =
  \frac{e^{t(2 - s)}}{2 - s}\biggr|_0^{\infty} = \frac{1}{s - 2}
\]
A justifiable question would be why don't we have \(\infty\) since
\(\lim_{t\to\infty}e^{t(2 - s)} = \infty\).
If this was the case, we would have a divergent integral.
In order for our integral to converge, we have to mind the exponential
term.
Let \(s = \sigma + i\omega\).
Then \(\exp[2t - \sigma]\exp[-i\omega]\).
\[
  \biggl\lvert\int_0^{\infty}\exp[(2 - \sigma)t]\exp[-it\omega]dt\biggr\rvert
  \leq\int_0^{\infty}\bigl\lvert\exp[(2 - \sigma)t]\bigr\rvert
  \bigl\lvert\exp[-it\omega]\bigr\rvert dt
\]
By Euler's formula, \(\exp[-it\omega] = \cos(t\omega) - i\sin(t\omega)\) so
\(\lvert e^{-i\omega}\rvert =
\sqrt{\cos^2(t\omega) + \sin^2(t\omega)} = 1\).
Thus, we have
\[
  \int_0^{\infty}\exp[t(2 - \sigma)]dt
\]
which converges when \(t(2 - \sigma) < 0\iff 2 < \sigma = \Re\{s\}\).
Therefore, since \(t(2 - \sigma) < 0\), we are justified in writing
\(-t(\sigma - 2)\) since \(\sigma - 2 > 0\).
We then have \(\mathcal{L}\{e^{2t}\}(s) = \frac{1}{s - 2}\).
\item
  \label{1c}
  \(f(t) = 2\cos(3t)\)
  \par\smallskip
  For this problem, we will need to use integration by parts twice.
  Let \(u = e^{-st}\) and \(dv = \cos(3t)dt\).
  Then \(du = -se^{-st}\) and \(v = \sin(3t)/3\)
  \begin{align*}
    \mathcal{L}\{2\cos(3t)\}(s)
    & = 2\int_0^{\infty}\cos(3t)e^{-st}dt\\
    & = \canceltoSwNe{\[\frac{2e^{-st}\sin(3t)}
      {3}\biggl|_0^{\infty}\]}{\(0\)} +
      \frac{2s}{3}\int_0^{\infty}\sin(3t)e^{-st}dt\\
    \intertext{For the second integration by parts, \(u\) and \(du\) will
    remain the same but \(dv = \sin(3t)dt\) so \(v = -\cos(3t)/3\).}
    & = \frac{2s}{3}\Biggl[\frac{-e^{-st}\cos(3t)}{3}\biggl|_0^{\infty} -
      \frac{s}{3}\int_0^{\infty}\cos(3t)e^{-st}dt\Biggr]\\
    (1 + s^2/9)2\int_0^{\infty}\cos(3t)e^{-st}dt & = \frac{2s}{9}\\
    \mathcal{L}\{2\cos(3t)\}(s) & = \frac{2s}{s^2 + 9}
  \end{align*}
\item
  \(f(t) = 1 - \cos(\omega t)\)
  \par\smallskip
  By \cref{1c}, we have
  \[
  \mathcal{L}\{1 - \cos(\omega t)\}(s) = \int_0^{\infty}e^{-st}dt -
  \int_0^{\infty}\cos(\omega t)e^{-st}dt =
  \frac{1}{s} - \frac{s}{s^2 + \omega^2}
\]
\item
  \(f(t) = te^{2t}\)
  \[
  \mathcal{L}\{te^{2t}\} = -\frac{\partial }{\partial s}
  \int_0^{\infty}e^{t(2 - s)}dt = -\frac{d}{ds}\frac{1}{s - 2} =
  \frac{1}{(s - 2)^2}
\]
where we require that \(\Re\{s\} > 2\).
\item
  \(f(t) = e^t\sin(t)\)
  \par\smallskip
  Again, we will use integration by parts where \(u = \sin(t)\) and
  \(dv = e^{t(1 - s)}dt\).
  \begin{align*}
    \mathcal{L}\{e^t\sin(t)\}
    & = \int_0^{\infty}\sin(t)e^{t(1 - s)}dt\\
    & = \canceltoSwNe{\[\frac{\sin(t)e^{t(1 - s)}}
      {1 - s}\biggr|_0^{\infty}\]}{\(0\)} +
      \frac{1}{s - 1}\int_0^{\infty}\cos(t)e^{t(1 - s)}dt\\
    & = \frac{1}{s - 1}\Biggl[
      \frac{\cos(t)e^{t(1 - s)}}{1 - s}\biggr|_0^{\infty} - \frac{1}{s - 1}
      \int_0^{\infty}\sin(t)e^{t(1 - s)}dt\Biggr]\\
    [1 + 1/(s + 1)^2]\int_0^{\infty}\sin(t)e^{t(1 - s)}dt
    & = \frac{1}{(s - 1)^2}\\
    \int_0^{\infty}\sin(t)e^{t(1 - s)}dt & = \frac{1}{(s - 1)^2 + 1}
  \end{align*}
\item
  \(f(t) =
  \begin{cases}
    1, & t\geq a\\
    0, & t < a
  \end{cases}\)
  \begin{align*}
    \mathcal{L}\{f(t)\}(s) & = \int_0^a0\cdot e^{-st}dt +
                             \int_a^{\infty}e^{-st}dt\\
                           & = \int_a^{\infty}e^{-st}dt\\
                           & = \frac{e^{-as}}{s}
  \end{align*}
\item
  \(f(t) =
  \begin{cases}
    \sin(\omega t), & 0 < t < \pi/\omega\\
    0, & \pi/\omega\leq t
  \end{cases}\)
  \par\smallskip
  This is just another exercise in integration by parts.
  \begin{align*}
    \mathcal{L}\{f(t)\}(s) & = \int_0^{\pi/\omega}\sin(t\omega)e^{-st}dt\\
                           & = \frac{\omega(1 + e^{-s\pi/\omega})}
                             {s^2 + \omega^2}
  \end{align*}
\item
  \(f(t) =
  \begin{cases}
    2, & t\leq 1\\
    e^t, & t > 1
  \end{cases}\)
  \begin{align*}
    \mathcal{L}\{f(t)\}(s) & = \int_0^1e^{-st}dt +
                             \int_1^{\infty}e^{t(1 - s)}dt\\
                           & = \frac{2(1 - e^{-s})}{s} +
                             \frac{e^{1 - s}}{s - 1}
  \end{align*}
  where we require that \(\Re\{s\} > 1\).
\end{exercise}
\item
  Compute the Laplace transform of the function \(f(t)\) whose graph is given
  in \cref{ch1sec1}.
  \begin{figure}[H]
    \centering
    \subcaptionbox{\label{ch1sec1prob2a}}{
      \includestandalone[width = 2in, mode = image]{Tikz/ch1sec1prob2a}}
    \qquad
    \subcaptionbox{\label{ch1sec1prob2b}}{
      \includestandalone[width = 3in, mode = image]{Tikz/ch1sec1prob2b}}
    \caption{Plots of \(f(t)\) for problem two.}
    \label{ch1sec1}
  \end{figure}
  The first step to these two problems is to determine the the function
  \(f(t)\).
  For \cref{ch1sec1prob2a}, we have
  \[
  f(t) =
  \begin{cases}
    1 - t, & 0 < t < 1\\
    0, & t > 1
  \end{cases}
  \]
  and for \cref{ch1sec1prob2b}, we have
  \[
  g(t) =
  \begin{cases}
    t, & 0 < t < 1\\
    2 - t, & 1 < t < 2\\
    0, & t > 2
  \end{cases}
  \]
  For \cref{ch1sec1prob2a}, the Laplace transform is
  \begin{align*}
    \mathcal{L}\{f(t)\}(s) & = \int_0^1(1 - t)e^{-st}dt\\
                           & = \frac{1 - e^{-s}}{s} +
                             \frac{\partial }{\partial s}\int_0^1e^{-st}dt\\
                           & = \frac{1 - e^{-s}}{s} + \frac{d}{ds}
                             \frac{1 - e^{-s}}{s}\\
                           & = \frac{s - 1 + e^{-s}}{s^2}
  \end{align*}
  For \cref{ch1sec1prob2b}, the Laplace transform is
  \begin{align*}
    \mathcal{L}\{g(t)\}(s) & = \int_0^1te^{-st}dt + \int_1^2(2 - t)e^{-st}dt\\
                           & = \frac{e^{-2s}(e^s - 1)^2}{s^2}
  \end{align*}
\end{exercise}

\section{Convergence}

\begin{exercise}
\item
  Suppose that \(f\) is a continuous function on \([0,\infty)\) and
  \(\lvert f(t)\rvert\leq M < \infty\) for \(0\leq t < \infty\).
  \begin{exercise}[label = (\alph*)]
  \item
    Show that the Laplace transform \(F(s) = \mathcal{L}\{f(t)\}(s)\) converges
    absolutely (and hence converges) for any \(s\) satisfying \(\Re\{s\} > 0\).
    \begin{align*}
      \mathcal{L}\{f(t)\}(s) & = \int_0^{\infty}f(t)e^{-st}dt\eqnumtag
                               \label{ch1sec2prob1aLT}\\
      \intertext{In order for the Laplace transform to converge absolutely,
      we need to show that
      \(\lvert\int_0^{\infty}f(t)e^{-st}dt\rvert < \infty\).}
                             & = \biggl\lvert\int_0^{\infty}f(t)e^{-st}dt
                               \biggr\rvert\\
                             & \leq M\int_0^{\infty}\lvert e^{-st}\rvert dt\\
      \intertext{Now, \(s = \sigma + i\omega\) where
      \(\sigma,\omega\in\mathbb{R}\).
      Then \(\lvert e^{-st}\rvert = e^{t\sigma}\lvert e^{-it\omega}\rvert\)
      since
      \(\lvert e^{-t\sigma}\rvert = \sqrt{e^{t^2\sigma^2}} = e^{t\sigma}\).
      Recall the trigonometric identity
      \(e^{i\theta} = \cos(\theta) + i\sin(\theta)\).
      Then \(\lvert e^{-it\omega}\rvert = \sqrt{\Re\{e^{-it\omega}\}^2 +
      \Im\{e^{-it\omega}\}^2} = 1\).}
                             & = M\int_0^{\infty}e^{t\sigma}dt\eqnumtag
                               \label{ch1sec2prob1a}
    \end{align*}
    \Cref{ch1sec2prob1a} converges when \(e^{t\sigma}\) is bounded.
    That is, \(t\sigma < 0\).
    From \cref{ch1sec2prob1aLT}, we see that boundedness occurs when
    \(-st < 0\) so \(st > 0\).
    Then \(-s < 0 \iff s > 0 \iff \Re\{s\} > 0\) where
    \(\Re\{s\} = \sigma\) since, by \cref{ch1sec2prob1a}, boundedness was due
    to \(\sigma\).
  \item
    Show that \(\mathcal{L}\{f(t)\}\) converges uniformly if
    \(\Re\{s\} > x_0 > 0\).
  \item
    Show that \(F(s) = \mathcal{L}\{f(t)\}(s)\to 0\) as \(\Re\{s\}\to\infty\).
  \end{exercise}
\item
  Let \(f(t) = e^t\) on \([0,\infty)\).
  \begin{exercise}[label = (\alph*)]
  \item
    Show that \(F(s) = \mathcal{L}\{e^t\}(s)\) converges for \(\Re\{s\} > 1\).
  \item
    Show that \(\mathcal{L}\{e^t\}(s)\) converges uniformly if
    \(\Re\{s\} > x_0 > 1\).
  \item
    Show that \(F(s) = \mathcal{L}\{e^t\}(s)\to 0\) as \(\Re\{s\}\to\infty\).
  \end{exercise}
\item
  Show that the Lpalace transform of the function \(f(t) = 1/t\), \(t > 0\)
  does not exist for any value of \(s\).
\end{exercise}

\section{Continuity Requirements}

\begin{exercise}
  \item
\end{exercise}
%%% Local Variables:
%%% mode: latex
%%% TeX-master: t
%%% End:
